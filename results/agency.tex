%Preamble
\documentclass[12pt]{article}
\usepackage{amssymb}
\usepackage{amsmath}
\usepackage{amsthm}
\usepackage{amsfonts}
\usepackage{commath}

\newtheorem{defn}{Definition}
\newtheorem{conj}{Conjecture}
\newtheorem{lem}{Lemma}
\newtheorem{thm}{Theorem}
\newtheorem{prf}{Proof}

%Body
\begin{document}

\title{Mathematical Agency and Computation Within a Universe}
\author{Ben Reis\footnote[\(\ast\)]{Imperfectly Educated Layman}}
\maketitle

\section{Introduction}
Much hay has been made over hypothetical potential of artificial general
intelligence. In particular, a number of organizations and researchers have come
to endorse the position that a "superintelligent" AI will be constructed at some
point in the near future, and that measures must be taken to ensure that said AI
is "friendly", ie not likely to cause significant harm to humanity as a whole.

The urgency of this task depends on the idea that "intelligence" is a
fundamentally unbounded property. That is, there is no maximally intelligent
mind and a mind (or some minds) capable of constructing other minds is capable
of constructing a mind that shares its goals but is more intelligent. The nature
of the task depends on the current nonexistence of any sort of superintelligent
AI and on the particular presumed form of any superintelligence that will emerge
in the future, namely that such an intelligence (and any successors it creates)
will be a single computer program, or possibly many copies of a single program,
implimented on a direct physical substrate.

However, at least as far as I can tell, relatively little work has been done to
check these assumptions. This is partially because of a difficulty in definition
- the fact that an agent must be embeded within the universe on which it acts
makes it difficult to rigorously define the problems an agent faces, and thence
prove limits on an agent's ability to solve them. In this paper I primarily wish
to define agency within a universe that has properties allowing the generated
results to apply to agents in the real world, without specifying the physics of
that world more than necessary.

\section{The Universe and its Properties}
\label{sec:universe}

We begin by defining the universe in which our agent will act. This will be
where most of our assumptions are located; the applicability of the results in
this paper depends on the defined universe matching up to our own as to the
effects of its properties on the capabilities of an agent embeded within it. I
will occasionally refer in this section to the agent or agents embeded within
the universe; these will be defined in Section \ref{sec:agents}.

The first and most fundamental assumption is that the universe is a universal
Turing machine or Turing equivalent construct - that is, a mathematical
construct capable of simulating Turing machines - but \textit{not} a more
powerful construct such as a hypercomputer, and not a construct with distinct
capabilities like a real computer\footnotemark. In the remainder of this
section, I will examine a classical Turing machine and a Turing complete
cellular automaton and determine how these classes must be restricted, if
necessary, to conform to our remaining axiomatic properties. In general I find
the cellular automaton's conformance to the axioms to be more
intuitive\footnotemark, but it may be easier to prove properties of agents with
respect to a Turing machine universe. The violation of these axioms will be
addressed briefly below but are not a focus of this paper.

\footnotetext{That the universe is not a real computer depends on the
holographic principle being a physical law. This is widely suspected but not
confirmed, and would forbid certain solutions to general relativity, the
implications of which are an unsolved problem in quantum gravity.}

\footnotetext{In general I find the intuitive conformance of cellular automata
to certain axioms of physics, and the existance of \textit{Day and Night}, a
Life-like rule that appears to exhibit a kind of charge symmetry, to be highly
suggestive in a physics context, but I lack the background to effectively
examine the implications. \cite{moriceau2011}, \cite{wacker2016}, and the works
they cite appear to be useful places to start when examining the idea of a
cellular automaton (perhaps somewhat generalized) that generates known laws of
physics.}

Second, we require that the universe has a property called \textit{locality}.
Locality requires that we introduce a notion of time and a notion of space.

To develop the idea of space, we split the universe into two groups of
components: the infinite components and the finite components. The infinite
components form the universe's \textit{space}, and the finite components form
the universe's \textit{physics}. In a Turing machine, the space is the Turing
machine's tape, and the state machine forms the physics. In a cellular
automaton, the cells form the space and the transition rule defines the physics.
In both cases, there is a finite list of cells which are considered to be
\textit{adjacent} to any given cell. These adjacency lists define the geometry
of the space, allowing us to construct spatial intervals - subsets of the
universe's space where cells and their adjacency lists form a connected graph.
We can also define the distance from one cell to another.
Note that in defining the space we have ignored the \textit{contents} of the
space - the state of the cells in the automaton or the symbols written on the
tape in the Turing machine.

%TODO write spatial interval and distance definitions in def environment

The universe's time can be defined in terms of the "execution" of the universe.
We take a static configuration of the universe, called the initial state - a
list of live cells in the automaton, or a tape position, starting state, and
collection of tape data in the Turing machine - and apply the rules of the
universe's physics once. We then say that time has advanced one step, and the
new configuration of the universe is called the result state.\footnote{This
discrete notion of time may or may not correspond to the physical reality we
inhabit, though it is expected to. Regardless, for reasons discussed in Section
\ref{sec:agents}, this should not affect the abstract capabilities of agents
embeded within the universe.} When we refer to the state of the universe "at" a
particular time step, we shall by convention be referring to the initial state
of that time step. We define a time interval \(\Delta t\) as a sequence of time
steps where for each step pair of steps \(t_{i-1}\), \(t_i\) in \(\Delta t\),
the initial state of step \(t_i\) is identical to the result state of step
\(t_{i-1}\).

%TODO write time interval definition(s) in def environment

%TODO discuss the arrow of time (maybe in footnote, maybe in next section)

We are now ready to define the property of locality:

\begin{defn}

   A universe obeys the principle of \textit{locality} iff for any finite time
   interval \(\Delta t\) beginning at initial time \(t_0\) and ending at final
   time \(t_f\) and any finite space interval \(\Delta x\) we can define a
   finite space interval \(\Delta y\) for which knowledge of the spatial state
   of \(\Delta y\), as well as any universal state, at \(t_0\), in addition to
   knowledge of the universe's physics, is sufficient to predict the state in
   \(\Delta x\) at \(t_f\).

\end{defn}

%TODO discuss relationship between locality and c.

%TODO discuss locality and c in TMs, automata.

Second, we assume that there is a maximum speed at which information
propagates, which we call \(c\), the computation speed of the universe. This
requires some notion of time in the universe. That is, we require that time pass
consistently in all parts of the universe. Cellular automata necessarily possess
this property, since every cell updates on every tick. Classical Turing Machines
lack universal time; they have local time, in that the read/write head performs
one operation on a cell each time the machine changes state, but a classical
Turing Machine can operate on a cell, perform arbitrary operations on a
different cell, and then propagate effects to the original cell. This allows
arbitrary events to occur between each tick local to the cell.

Third, we assume that the universe is operating on random data for a time
interval \(t_\textit{hist} \gg t_l\), where \(t_l\) is the time interval over
which the agent under consideration has been operating.

Finally, we can without loss of generality disregard any halting state of the
universe, since the halting of the universe necessarily halts the agent or
agents embeded within.

\section{Agency}
\label{sec:agents}

% see: Incomplete Models

\section{Intelligence, Naturalized Induction, and Interaction}
\label{sec:intelligence}

%TODO see: Problems Tackled 2.1, 2.2; Omohundro Drives; Soars World Models

\section{Real-Time Strategy}
\label{sec:games}


\section{Personhood}
\label{sec:persons}


\section{Composite and Implicit Agents}
\label{sec:composites}


\bibliographystyle{plain}
\bibliography{agency}
\end{document}
